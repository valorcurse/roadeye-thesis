\hoofdstuk{Inleiding}

Even nowadays, it is still difficult for law enforcement to follow up on every case of stolen property, e.g. cars, simply because they cannot be everywhere at once. A possible way to improve law enforcement's efficiency would be by including the aid of civillians through the voluntary provision of gathered information from one's environment. This used to be a very difficult issue because required people to carry dedicated, and often large, hardware devices which made the gathering of information possible. Since the breaking of the smartphone age, a large number of people carry a computer in their pockets. This internet-connected all-purpose device allows for a whole range of new possibilities and with the right software, gives the user the possibility to help his own community and therefore make it a better place. The objective of this project is to create software which will give users the ability to search for stolen cars, using their smartphone's camera and license plate information made available by the police.

% De inleiding bevat informatie die de lezer moet weten om de rest van
% het verslag te kunnen lezen. Het is niet goed als de lezer pas
% halverwege het verslag er achter komt waar het over gaat. De inleiding
% bevat daarom het doel en de context van de afstudeeropdracht en -- in
% bijzondere gevallen -- de opzet van het verslag.


% Het doel van de afstudeeropdracht betreft het maken\footnote{Maken in
% brede zin, dus ook het ontwerpen.} van een product of het leveren van
% een dienst of het uitvoeren van een onderzoek of een combinatie
% daarvan. Het doel wordt ook duidelijker als de problematiek die tot de
% opdracht geleid heeft, wordt beschreven. Zonder beschrijving van de
% opdracht en de achtergrond van de opdracht is het doel onbekend en is
% de rest van het verslag onduidelijk.


% Het aansluitende hoofdverslag bevat het verhaal van functionele en
% architectonische keuzes en de onderbouwingen daarvan. De keuzes die
% gemaakt zijn, kunnen alleen verklaard worden als er sprake is van een
% context en doel. Zij bepalen direct of indirect de eisen die aan de
% functionaliteit, de ergonomie, de prestatie, de veiligheidsrisico's en
% de bedrijfszekerheid gesteld moeten worden.  De context is dus het
% maaiveld waarop de keuzes worden onderbouwd. Dit maaiveld mag niet te
% klein, maar ook niet te groot zijn.


% Uitleg over de opzet van het verslag in de inleiding is noodzakelijk
% als de inhoudsopgave ontbreekt en als het project een ingewikkelde
% structuur heeft, bijvoorbeeld een combinatie van deelprojecten met
% ontwerp- en onderzoeksactiviteiten. Je zou dan kunnen aangeven waarom
% bepaalde zaken in de bijlagen geplaatst zijn. In het algemeen is een
% ingewikkelde structuur lastig te overzien en wordt een
% rechttoe-rechtaan verhaal beter begrepen. Plaats om die reden alles
% wat de leesbaarheid verlaagt maar wel noodzakelijk is voor de
% onderbouwing en de bewijsvoering in de bijlagen.

