\hoofdstuk{Introduction}

This project has been made available by CGI Nederland BV, which is a daughter company of the CGI Group Inc. located in The Netherlands. CGI is a multinational IT (Information Technology) company which provides consulting, systems integration, outsourcing, etc. on a multitude of different technological areas. Because of its affinity with these different kinds of technology, CGI researches various technologies to solve problems facing modern society. As a result of this drive, the idea for this project arose: How can people be given the ability to help track stolen cars, and by extension improve their community? Since everyone nowadays carries a smartphone around in their pocket, there is a lot of untapped processing power which could be used for this purpose. This led to the invitation of a student to research and implement this project as the topic for a bachelor thesis. I chose this project because of its relation with Computer Vision, a computer science topic I have worked with before and find interesting.

\paragraaf{Problem}

Even nowadays, it is still difficult for law enforcement to follow up on every case of stolen property, e.g. cars, simply because they cannot be everywhere at once. A possible way to improve law enforcement's efficiency would be by including the aid of civilians through the voluntary provision of gathered information from one's environment. This used to be a very difficult issue because it required people to carry dedicated, and often large, hardware devices which made the gathering of information possible. Since the breaking of the smartphone age, a large number of people carry a computer in their pockets. This Internet-connected all-purpose device allows for a whole range of new possibilities and with the right software, gives the user the ability to help his own community and therefore make it a better place. By using the smartphone's camera to capture video images, that information can then be paired with a Computer Vision algorithm to give the user ability to help find stolen cars. This project will therefore be focused on applying the Computer Vision discipline to solve this problem and try to answer the following question: \textit{How can license plate information be gathered from images of a smartphone camera using software?}

But this only concerns the problem of this project in a more general way, without taking any requirements or conditions into account. Because the software must run on a mobile platform it must take all the limitations of such a platform or the environment where it will operate into consideration, i.e. limited processing power, battery life, unstable images or car distance. To keep these conditions in mind while developing the application, the following questions will be answered: \textit{`How can the software be optimized to work in a correct way from within a mobile device?'} and \textit{`What are the limitations of such an application?'}.

\paragraaf{Objective}

The objective of this project consists in creating an Android smartphone software application that is able to locate license plates in video images gathered from the smartphone's camera. From these images, the application must be capable of reading the alphanumeric text displayed on the plates. This information will then be compared to a list of license plate information, which was fetched from an website beforehand. If there is a match, the application must inform a central application of the said match, along with its position when the image was captured and how reliable the recognition is.

\paragraaf{Scope}

The scope of this project will encompass the development of the Android application which will recognize license plates using the images from the phone's camera, the communication with the web page where the license plate information will be fetched from and the communication with the main application where information over the recognition will be sent to. It will only take into account Dutch rectangular yellow car license plates and might therefore not work with foreign plates. It will take into account, implementation and design wise, privacy concerns according to the Dutch law. The software will only be written and tested for Android version 4.2.2, running on an HTC One X.
