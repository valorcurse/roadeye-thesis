\hoofdstuk{Hoofdverslag}

In het hoofdverslag moet een rode draad aanwezig zijn,
zodanig dat het een leesbaar artikel wordt. Deze leesbaarheid houdt in
dat het publiceerbaar moet zijn in een vakblad. Dit betekent dat voor
de onderbouwing noodzakelijke gegevens (tabellen, grafieken en
tekeningen) die de leesbaarheid kunnen verlagen, worden opgenomen in
bijlagen. Bijlagen worden niet gepubliceerd.


Plaatjes worden geplaatst als de tekst te ingewikkeld wordt.
Plaatjes moeten wel passen in de rode draad. Ook worden aan plaatjes
`leesbaarheidseisen' gesteld. Haal van te voren overbodige
informatie uit de plaatjes. Afstudeerverslagen met een grote
hoeveelheid `witruimte', `screendumps' en `clipart' zijn verdacht,
meestal duidt dit op een gebrek aan inhoudelijke informatie.


De redactie van het hoofdverslag betreft niet alleen het opmaken van
de teksten, het verwijderen van spelfouten, het controleren van
verwijzingen en het nummeren van de pagina's, maar ook het zoveel
mogelijk verwijderen van opsommingen (bullets). Hoewel opsommingen bij
het opstellen van een teksten in eerste instantie als handige
kapstokken fungeren, moeten zij uiteindelijk als gewone alinea's
of paragrafen herschreven worden.


Het verschil tussen eigen teksten en teksten van anderen moet
duidelijk worden aangegeven. Bronvermelding is verplicht, het weglaten
daarvan kan duiden op plagiaat. Een goed hoofdverslag bevat meestal
verwijzingen naar relevante en recente vakliteratuur en publicaties
van gerenommeerde onderzoeks- en onderwijsinstellingen.  Een
afstudeerverslag van een bacheloropleiding moet verplicht verwijzen
naar recente en relevante vakliteratuur. Indien een afstudeerverslag
hieraan niet voldoet, wordt het afgekeurd.


Het hoofdverslag is een verhaal. De inhoudelijke beoordeling van dit
verhaal is bij ingenieursopleidingen gebaseerd op de verantwoording
van de `keuzes' die gemaakt zijn. De argumentatie van deze gemaakte
keuzes is direct gebaseerd op de vastgestelde functionele eisen (het
pakket van eisen), de randvoorwaarden betreffende de kwaliteit en
risico's en indirect gebaseerd op de context.


Niet alleen goede maar ook de verkeerde keuzes die gemaakt zijn, zijn
belangrijk voor de beoordeling. Vaak blijken verkeerde keuzes tot
interessante conclusies te leiden. Ook zijn ze noodzakelijk om een
indruk te krijgen van de inspanning die de student heeft
geleverd. Naast de keuzes tijdens het ontwerp zijn keuzes voor
hulpmiddelen en gereedschappen relevant voor de beoordeling.  Als de
student bijvoorbeeld zelf een ontwerpmethode gekozen heeft, dan zal
die keuze verantwoord moeten worden. Als de methode voorgeschreven is
in de randvoorwaarden van de projectopdracht, dan zal aangetoond
moeten worden dat de methode op correcte manier is toegepast.


Om een goed verhaal op te stellen, moet vooraf aan enkele voorwaarden
worden voldaan. De eerste voorwaarde is de geschiktheid van het
afstudeerproject. Als een afstudeerproject niet tot keuzes leidt, kan
men zich afvragen of dat wel een echte afstudeeropdracht is. Een
afstudeerproject zonder onderzoeksaspecten is ook verdacht. Daarnaast
moet een afstudeerproject passen in het profiel van een opleiding om
beoordeelbaar te zijn. De andere voorwaarde voor goed een verhaal is
de registratie van werkzaamheden tijdens het afstudeerproject. Dit
voorkomt dat men dingen vergeet.


Laat je verslag lezen door een niet bij het afstudeerproject betrokken
persoon. Meestal ziet die al snel waarin je verslag tekort schiet.


\paragraaf{Werken met \LaTeX{}}

Het is niet verplicht om met \LaTeX{} te werken. Men mag ook gebruik
maken van andere tekstverwerkers zoals \emph{MS-Word}, Wel is het
verplicht het afstudeerverslag \LaTeX{}-geformateerd in te leveren en
van de \LaTeX{}-template \verb!modelverslag.sty! gebruik te
maken.

De \LaTeX{}-template bevat enkele macro's voor het opstellen van een
hoofdstuk (\verb!\hoofdstuk!), een paragraaf (\verb!\paragraaf!), een
afbeelding (\verb!\figuur!). De overige \LaTeX{} macro's en omgevingen
blijven bruikbaar. Bijvoorbeeld de \verb!tabular!-omgeving om tabellen
te maken:

\begin{lstlisting}{language=[LaTeX]TeX}
\begin{tabular}{formaat}
   ... 
\end{tabular}
\end{lstlisting}

\begin{center}
\begin{tabular}{|l||r|}
  \hline
  \multicolumn{2}{|c|}{Afmetingen ($1\mbox{ pt}=0,351\mbox{ mm}$)}\\
  \hline
  paper width     & \the\paperwidth\\
  text width      & \the\textwidth\\
  column width    & \the\columnwidth\\
  column seperate & \the\columnsep\\
  oddside margin  & \the\oddsidemargin\\
  evenside margin & \the\evensidemargin\\
  paper height    & \the\paperheight\\
  text height     & \the\textheight\\
  top margin      & \the\topmargin\\
  \hline
\end{tabular}
\end{center}

Een nadeel van tabellen dat ze vaak te groot zijn voor de
twocolumn-mode. Het zou mooi zijn als ze ingedrukt kunnen
worden. Bovendien is deze tabel niet-zwevend, hij wordt geplaatst
tussen de tekstdelen waar hij is ingevoerd. Dit kan bezwaarlijk zijn
bij pagina-overgangen. In dat geval kan je beter gebruikmaken van
zwevende tabellen (en figuren) die door \LaTeX{} zelf op een geschikte
plaats worden gezet. Wel moet aan een zwevende tabel een label en een
onderschrift gekoppeld worden om er naar te kunnen verwijzen. Voor een
zwevende horizontale tabel met label en onderschrift wordt in de
`template' de \verb!tabel!-omgeving aangeboden:\\

\begin{Aanpassen}
\begin{verbatim}
\begin{tabel}[afm]{formaat}{label}{onderschrift}
  ...
\end{tabel}
\end{verbatim}
\end{Aanpassen}


De \verb!tabel!-omgeving plaatst `zwevende' tabellen in verslag- en
publicatie-mode. Het eerste argument is een optioneel \verb![afm]!
argument met de defaultwaarde \verb!\normalsize! voor de afmeting van
de karakters. De mogelijke waarden voor de afmeting zijn -- van groot
tot klein -- de volgende macro's: (\verb!\huge!, \verb!\LARGE!,
\verb!\Large!, \verb!\large!, \verb!\small!, \verb!\footnotesize!,
\verb!\scriptsize! en \verb!\tiny!).

Bovendien zijn de standaard \verb!tabular! kolomformaten
\verb!r,l,c,|,||,p{lengte}! uit de tabelomgeving uitgebreid met
kolomformaten \verb!\R, \C, \L!  voor variabele celinhoud zoals het
plaatsen van meerdere regels per cel.

Een verticale tabel is mogelijk met de omgeving (\verb!TABEL!)  met
dezelfde kolomformaten mogelijkheden.  In \LaTeX{} zijn de tabellen,
vooral in de \verb!twocolumn!-mode erg lastig. Bijvoorbeeld in de
tabellen~\ref{tab:vbt} en \ref{tab:vbx} zijn twee verschillende
uitwerkingen van de tabelomgevingen:

\begin{footnotesize}
\begin{tabel}[\Large]{|r|l|}{vbt}{Vaste cellen, variabele breedte}
  \hline
  7C0 & hexadecimal \\
  3700 & octal \\ \cline{2-2}
  11111000000 & binary \\
  \hline \hline
  1984 & decimal \\
  \hline
\end{tabel}
\end{footnotesize}

%Absolute breedte in mm, cm etc. (geschikt voor één mode):
%\begin{tabel}{|>\R p{2cm}|>\L p{4cm}|}{vbx}{OpenGL libraries}
%Relatieve breedte 20% en 65% van de kolombreedte (geschikt voor alle moden).
%Procentwaarden van 0 ... 99, voor 100% neem \columnwidth zelf.
\begin{tabel}{|>\R p{\Procent{20}}|>\L
    p{\Procent{65}}|}{vbx}{Variabele cellen, variabele breedte}
\hline
 OpenGL core library & OpenGL32 voor MS-Windows en GL voor
 de meeste X-Window systemen\\
\hline
 OpenGL Utility Library & GLU\\
\hline
 Koppeling met het platform & GLX voor X-Window en WGL voor MS-Windows\\
\hline 
 OpenGL Library Utility Toolkit & GLUT, bibliotheek voor
  het openen van windows, invoer van muis en toetsenbord, menus,
  event-driven in- en uitvoer\\
\hline
\end{tabel}

Plaats afbeeldingen alleen in het hoofdverslag als ze de tekst
ondersteunen en de leesbaarheid niet verlagen.  In de tekst kan naar
afbeeldingen worden verwezen met de macro \verb!\ref{fig:label}!.

In \LaTeX{}\cite{lam1994} geschreven verslagen zijn op diverse manieren
afbeeldingen\cite{Oos1996} te plaatsen. Een van die manieren is gebruik te
maken van de macro \verb!\figuur! in de \verb!modelverslag!-package'.

`Vector graphics' figuren van het `pdf-', `eps-' en `svg-'
formaat\footnote{Een pdf-bestand kan zowel vector-graphics als
  bitmap-graphics bevatten.} met een ingewikkelde `bounding box' zijn
moeilijk op de juiste schaal te brengen. Vaak moet dat met uitproberen
bepaald worden. Het plaatsen van figuren met absolute afmetingen of
een vaste `scale' factor, kan leiden tot minder soepele oplossingen
zoals figuur~\ref{fig:PDFA}. Deze figuur heeft naast een rotatie
(\verb!angle=270!)  een vaste scale-factor (\verb!scale=0.45!) die
alleen geschikt is voor de `twocolumn-mode'.

\begin{center}
  \figuur{scale=0.45,angle=270}{plaatjes/agp.pdf}{PDFA}{Vaste breedte
    (pdf)}
\end{center}

In plaats van \verb!scale=x! kan je beter de relatieve afmeting
\verb!width=\Procent{y}! gebruiken. De waarde $y$ wordt in de
verslag-mode met uitproberen gevonden, zie figuur~\ref{fig:PDFR}.

\figuur{width=\Procent{40},angle=270}{plaatjes/agp.pdf}{PDFR}{Variabele
  breedte (pdf)}

Het afmetingsprobleem is iets gemakkelijker op te lossen met `bitmap
graphics' van het `jpg-', `gif-' en `png-' formaat omdat de figuren al
van te voren geschaald kunnen worden als de `bounding box' bij het
inlezen bekend is. De breedte (\verb!width!) kan als percentage van de
kolombreedte (\verb!width=\Procent{0 ... 99}!) worden opgegeven zoals
dat bij figuur~\ref{fig:PNGR} gedaan is. Voor een 100\% waarde neemt
men \verb!width=\columnwidth! De afmeting wordt automatisch
aangepast aan de nieuwe kolombreedte.


\begin{center}
\figuur{width=\columnwidth}{plaatjes/agp.png}{PNGR}{Variabele breedte (png)}
\end{center}


De macro \verb!\PROCENT{0...99}! is nodig voor de macro's \verb!Tabel!
en \verb!Figuur!. Deze laatste twee macro's maken het mogelijk dat
tabellen en afbeeldingen in de twocolumn-mode passen met behoud van
hun originele afmeting en detaillering (zie
figuur \ref{fig:FIXED}). De parameters van deze macro's komen overeen
met de parameters van de macro's \verb!tabel! en \verb!figuur!.

\Figuur{width=\PROCENT{40},angle=270}{plaatjes/agp.pdf}{FIXED}{Vaste breedte ook in twocolumn-mode (pdf)}

In het algemeen heeft vector-graphics een betere kwaliteit van de
weergave dan bitmap-graphics.


\paragraaf{Bijzondere tekens en afbreekproblemen}

Bijzondere tekens zoals de á, à, ä, é, è, ë, ï, ü, ç \ldots worden
probleemloos door \LaTeX{} geaccepteerd als normale utf8
karakters. Voor de uitzonderingen bestaan macro's zoals het
euro-symbool \euro{} waarvoor de macro \verb!\euro! nodig is. In
wiskundige formules kan je gebruik maken van de macro \verb!\eurom!.


In de two-columnmode zijn regels soms te lang als er gebruik gemaakt
is van \verb!verb! of \verb!verbatim! of woorden die niet goed worden
afgebroken. In dat laatste geval kan je in zo'n woord een afbreekpunt
introduceren met de twee tekens \verb!\-!. Een regel kan gecontroleerd
afgebroken door van te voren onzichtbare knikpunten te plaatsen met de
\verb!\Knak! macro. De volgende regel moet in in tegenstelling met de
twocolumnmode in de verslagmode ongeknakt worden weergegeven:

\begin{Aanpassen}
\begin{verbatim}
 ... aaaaaaa\Knak{}aaaaaaa ...
\end{verbatim}
\end{Aanpassen}


aaaaaaaaaaaaaaaaaaaaaaaaaaaaaaaaaaaaaaa\Knak{}aaaaaaaaaaaaaaaaaaaaaaaaaaaaaaaaaaaaaaa.

Voor regels waarbij de structuur niet gebroken mag worden, is de
\verb!\Knak!-methode ongeschikt, bijvoorbeeld bij scripts en
broncode. Daarentegen zorgt de \verb!Aanpassen!-omgeving ervoor dat in
de twocolumn-mode de regels met behoud van de originele structuur
worden weergegeven. Daarvoor wordt een kleinere letterafmeting
gebruikt (default de \verb!\scriptsize!). Deze omgeving werkt alleen
met niet al te lange regels. Bij zeer lange regels moet de
letterafmeting zeer klein worden waardoor de leesbaarheid in het
gedrang komt. In dat geval moet naar een andere oplossing gezocht
worden zoals het opnemen van de probleemregels (broncode en scripts)
in de bijlagen.

\begin{Aanpassen}[\tiny]
aaaaaaaaaaaaaaaaaaaaaaaaaaaaaaaaaaaaaaaaaaaaaaaaaaaaaaaaaaaaaaaaaaaaaaaaaaaaaa.
\end{Aanpassen}

Hoewel het gebruik van opsommingen (\verb!\item!), letterlijke citaten
\verb!quotation! en kaders (\verb!\fbox!) in de twocolumn-mode tot
problemen kunnen leiden, zijn ze beperkt toegestaan. Bijvoorbeeld voor
de kaders rond de teksten kan je beter gebruik maken van de
\verb!tabular!-omgeving (of de \verb!tabel!-omgeving als je geen last
wil hebben van pagina-overgangen), dan voor de standaard
\verb!\fbox!-methode. De kolom van deze omkaderde tabel moeten dan wel
een relatieve afmetingsverhouding de \verb!\columnwidth! krijgen.

\begin{Aanpassen}
\begin{verbatim}
\begin{center}
\begin{tabular}{|>\C p{\Procent{80}}|}
  \hline
  Afbreekproblemen ...
  \hline
\end{tabular}
\end{center}
\end{verbatim}
\end{Aanpassen}

\begin{center}
\begin{tabular}{|>\C p{\Procent{80}}|}
  \hline
  ~\\
  Afbreek- en andere opmaakproblemen pak je als laatste aan,
  dus bij je definitieve verslag!\\
  ~\\
  Tabellen, figuren en listingen in het hoofdverslag tot het
  noodzakelijke beperken.\\
  ~\\
  \hline
\end{tabular}
\end{center}


\paragraaf{Algoritmen en broncode\cite{wikibooks}}

Als je algoritmen met een mooie layout wilt hebben, dan zou je het
\verb!algorithmic!-pakket kunnen gebruiken. Met dit pakket kan je het
algoritme op een logische manier opbouwen met pseudotaal. Het bestand
`verslag.tex' bevat al de pakketten \verb!algorithmic! en
\verb!listings! die voor dit verslag nodig zijn. Als je zelf packages
wil toevoegen of verwijderen (afblijven van
\verb!\usepackage{moduleverslag}!)  dan moet dat in de preambule
`verslag.tex'.

\begin{Aanpassen}
\begin{verbatim}
\usepackage{algorithmic}
\end{verbatim}
\end{Aanpassen}

Een algoritme moet je maken binnen een algorithmic-omgeving, een
voorbeeld:

\begin{Aanpassen}[\small]
\begin{algorithmic}
\IF {$i\geq maxval$} 
        \STATE $i\gets 0$
\ELSE
        \IF {$i+k\leq maxval$}
                \STATE $i\gets i+k$
        \ENDIF
\ENDIF 
\end{algorithmic}
\end{Aanpassen}


Broncode kan je in een \verb!verbatim!-omgeving opnemen. De
broncoderegels zien er net zo uit zoals je ze ingetypt hebt.  Het
\verb!listings!-pakket is geavanceerder dan de
\verb!verbatim!-omgeving.

\begin{Aanpassen}
\begin{verbatim}
\usepackage{listings}
\end{verbatim}
\end{Aanpassen}

Merk even op dat alle commando's van het \verb!listings!-pakket
beginnen met \verb!lst!, dit conform de lppl-licentie.

De broncode zelf zet je in een \verb!listings!-omgeving, net zoals bij
de \verb!verbatim!-omgeving, om broncode te zetten gebruik je het
\verb!\lstinline!-commando op dezelfde manier als het
\verb!\verb!-commando. Je kunt ook broncode van een extern document laden met het commando:

\begin{Aanpassen}
\begin{verbatim}
\lstinputlisting{pathname}
\end{verbatim}
\end{Aanpassen}

Het argument `pathname' is de relatieve of absolute locatie van het
bronbestand, de map(pen) gecombineerd met de bestandsnaam. Als je
broncode van een bronbestand laadt, ben je zeker dat de broncode in je
\LaTeX{}-document altijd actueel is en hou je het \LaTeX{}-document
overzichtelijk. Als de broncode niet in dezelfde map of een submap van
het \LaTeX{}-document staat of je gebruikt absolute `pathnames', dan
is het mogelijk dat het verslag niet op andere computers gecompileerd
kan worden. Bij het inleveren van je afstudeerverslag in
\LaTeX{}-formaat zal je hiermee rekening moeten houden.


Alle opties in het \verb!listings!-pakket hebben eenzelfde structuur
\verb!sleutel=waarde!. Als je alleen 'Java' gebruikt hebt, dan kan je
deze taal voor je volledig document na de regel
\verb!\usepackage{listings}! in preambule `verslag.tex' definiëren met
\verb!\lstset{language=java}!

\lstset{language=java}

\begin{Aanpassen}
\begin{lstlisting}
public class HelloWorld {
    public static void main(String[] args) {
        System.out.println("Hello, world!");
    }
}
\end{lstlisting}
\end{Aanpassen}


De sleutel is hier dus \verb!language! en de waarde die je aan de
sleutel geeft is \verb!java!. Alles wat je als opties binnen de
\verb!\lstset!-macro zet kan je per \verb!listings!-omgeving apart
definiëren. Bijvoorbeeld html-broncode met
\verb!\begin{lstlisting}[language=html]!:

\begin{Aanpassen}
\begin{lstlisting}[language=html]
<html>
    <head>
        <title>Hello</title>
    </head>
    <body>Hello</body>
</html>
\end{lstlisting}
\end{Aanpassen}

