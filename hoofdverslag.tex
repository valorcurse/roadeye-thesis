\hoofdstuk {Methodology}

<insert text here>

\hoofdstuk{Thesis}

\paragraaf{Algorithms}

When searching for possible algorithms with the functionality to find license plates in an image, two main types came forth from the research: feature detection and edge detection. 

The feature detection algorithms work by finding so called \emph{features} in a image, which are used to recognize the first image in a second one. These features are segments of an image which must be uncommon, as to reduce the possibility of retrieving a false positive when applying the algorithm, and also consist of something which can be objectively described to a computer. Because of these requirements, the features extracted from an images are usually corners since a corner only matches itself if compared to other segments of the image, as opposed to flat surfaces or lines which may appear in multiple times in multiple places. Because this algorithm focuses on detecting the uniqueness of an image and using that attribute to detect it in different images, it becomes difficult to use feature detection to recognize license plates because every license plate contains unique text. These false positives originate from the diversity in shapes that exist in the latin alphabet and this considerable collection features is often detected in random and incorrect locations. One possible approach to use this algorithm to find the plate location would be by creating a feature database of every possible alphanumeric character and then finding the highest concentration of text as a possible location. Due to little information on the performance of this algorithm, the other algorithm was chosen.

The other possible algorithm is mostly based on edge detection. This kind of algorithm works by applying an edge detection algorithm to a greyscale version of the image where the car is present, e.g. the Sobel Filter [ref here] or Canny Edge Filter [ref here]. This creates a black and white image where the edges of every object in the images are displayed in white. One of the characteristics of a license plate is the presence of a high amount of edges due to the text displayed on it. This means that by using this algorithm it is possible to find the location of the license plate by looking for the highest edge density area in the image. This method of plate location has been chosen because of the large amount of information that proves its efficiency and performance.

\paragraaf{Architecture}








\paragraaf{Implementation}

% for detecting vertical edges, it is possible generate a graph on the amount of edges in a specific row by summing all the values on that same row together. One of the characteristics of a license plate is the presence of a high amount of edges due to the text displayed on it. This makes it possible to find the vertical location of the plate by looking for the highest peak on the edges graph and using the left and right base of that peak as the delimiter for the location of the license plate. Because there might be an object in the image with a higher edge density than the plate, a pre-defined number of candidates are detected by the algorithm (default is 3) for increasing the chance of finding the correct location. The results in a number of horizontal segments of the original images, called bands. 

% therefore the features detected in the image create a lot of false positives.   

% \newpage

% \paragraaf{Implementation}
