\samenvatting

Even nowadays, it is still difficult for law enforcement to follow up on every case of stolen property, e.g. cars, simply because they cannot be everywhere at once. A possible way to improve law enforcement's efficiency would be by including the aid of civillians through the voluntary provision of gathered information from one's environment. This used to be a very difficult issue because required people to carry dedicated, and often large, hardware devices which made the gathering of information possible. Since the breaking of the smartphone age, a large number of people carry a computer in their pockets. This internet-connected all-purpose device allows for a whole range of new possibilities and with the right software, gives the user the possibility to help his own community and therefore make it a better place. The objective of this project is to create software which will give users the ability to search for stolen cars, using their smartphone's camera and license plate information made available by the police.

% Het doel van een samenvatting is om potentiële lezers zo snel mogelijk
% te overtuigen van de relevantie van het verslag. Als afstudeerverslagen
% gepubliceerd worden, is digitaal zoeken noodzakelijk. Daarom worden in
% de samenvatting (\emph{abstract}) vaak kenmerkende woorden en
% uitspraken opgenomen. Een samenvatting voor een afstudeerverslag mag
% niet meer dan een paar honderd woorden bevatten.
