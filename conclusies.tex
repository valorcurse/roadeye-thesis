\hoofdstuk{Conclusion}

At the start of this thesis the following questions were asked: `How can license plate information be gathered from the images of a smartphone camera using software?', `How can the software be optimized to work in a correct way from within a mobile device?' and `What are the limitations of such an application?'. Using the information gathered during the course of the project, these questions will now be answered.

\paragraaf{How can license plate information be gathered from the images of a smartphone camera using software?}

Because this project was focused only on Dutch yellow rectangular car license plates, the algorithm was therefore specifically created for that purpose. The first step is to remove everything in the image that is not yellow. This leaves a binary image where in most cases a clear area where the plate is present. Yet because there are yellow cars or other objects with large yellow areas, before grabbing that area as the plate's location, an edge detection algorithm is applied to the the binary image to retrieve only the edges of the objects. The reason for this is to prevent those large yellow areas and to increase the change of finding the correct area by using one characteristic of a license plate, a high edge density thanks to the presence of text on the plate. Once the edges images has been created, all the values are summed up per row. This creates a graph with which the vertical location of the plate can be found by searching for the broadest group of peaks. By applying this method to the three broadest group of peaks present in the graph, three possible vertical locations of the license plate are found, also known as `bands'. The reason for multiple bands is to increase the change of finding the plate. By applying the same method but now for every column on all bands and applying some constraint filters to the possible plate, the full location of the plate is found. It's possible the plate is rotated depending on the angle the image was captured at. To solve this, the angle of the plate is calculated and transformed. The correct plate has a contour finding algorithm applied to, so the characters within the plate can be retrieved. This image can then be given to the text recognition software to parse the text.

\paragraaf{How can the software be optimized to work in a correct way from within a mobile device?}

Because smartphones are limited devices when it comes to processing power, when creating such an application these limitations must be taken into account. This was solved using two different approaches: multi-threading and buffers.

Starting with multi-threading, this technique allows an application to share its processing load with every processor core in the smartphone. The application uses four threads for its main four components: the UI thread, the band localization thread, the plate localization thread and the text recognition thread. This allows for, in the case of the smartphone used to test the application, a dedicated core for every thread. Every time there is a new item available in one of the buffers, the respective thread is executed with the new data.

The buffers approach takes into consideration that a smartphone does not have enough processing power to process images at real-time. The outputs of every thread, with the exception of the text recognition thread because its output is handled immediately, are therefore stored within a buffer so they can be processed in their own time. There are 3 different buffers: the frames buffer, the bands buffer, and the plates buffer. 
The frames buffer stores the video frames capture with the smartphone's camera and has a maximum size of 1. This prevents the accumulation of frames which were captured shortly after each other which leads to a high chance of processing the same plate multiple times. The other two buffers store bands and plates respectively and have no size limitations because when the initial frame has been processed, the follow-up data is considered important and is processed evenly.

Using this two methods it's possible to create a functional application despite its need for a large amount of processing power.


\paragraaf{What are the limitations of such an application?}

Because of the nature of this application, it brings a couple of problems to its usability. One of these problems is the distance at which the application can still process a plate reliably. After performing tests to find this maximum distance, the results indicated the application was still able to provide reliable results at the distance of 5 meters, while at the distance of 6 meters it was still able to sometimes provide correct results but not in a reliable manner. Another problem is the angle at which the results are still processed correctly. Tests demonstrated that the maximum angle is located at around 47$^{\circ}$ at a distance of 4 meters. Using this information, an area can be derived where the application works well and where it starts having problems. This area is depicted in Figure \ref{fig:road-situation} and the scenario was created using the road design criteria from the province of Zuid-Holland \cite{road-design}. Looking at this figure it's clear that the application can process the plates of vehicles both in the same lane and in the adjacent lanes while driving at a safe distance. Further, the text recognition component does not work flawlessly because of the large number of possible combinations of angles, distances, light conditions, and obstructions that plates in captured images are subject to and in turn results in incorrect text parsing. However this isn't a large problem for the application because its purpose is to find a predefined set of license plates, which means the application only needs to parse the plate correctly once and therefore a large number of false positives has little impact in the reliability of the results.

\begin{figure}[ht]
    \centering
    \includegraphics[width=0.3\textwidth]{plaatjes/roadeye-road}
    \caption{Application flow diagram.}
    \label{fig:road-situation}
\end{figure}%


\hoofdstuk{Recommendations}

\paragraaf{Distance and speed}

The distance and speed at which the application is able to capture images is entirely dependent on the hardware of the smartphone. A better camera allows for either a very high resolution to be used at a decent frame rate, which would improve the distance and the quality of the text recognition, or a lower resolution with a higher frame rate, which would allow the capture of plates from cars which are driving in the opposite direction. Beware that with a higher resolution, the processing time of the application would rise significantly.

\paragraaf{Angle}

To improve the angle at which the application is able to retrieve the characters correctly, aside from a better camera, a Principal Component Analyser (PCA) algorithm could be applied. This kind of algorithm is used to find a linear pattern within a dataset and if that dataset happens to be a character, it calculates the direction it is ``pointing'' to. Using that information the angle a which the character is slanted can be calculated. Using the calculate angle, a shear transform can then be applied to remove the slant of a character, as displayed in Figure \ref{fig:slanted-plate} and \ref{fig:unslanted-plate}. Although it works most fine most of the times, there are some characters from which the angle is incorrectly calculated and results worse off than before, e.g. \ref{fig:not-slanted-plate} and \ref{fig:incorrectly-unslanted-plate}. To solve this the algorithm must be modified to work better with characters.

\begin{figure}[ht]
        \centering
        \begin{subfigure}{0.33\textwidth}
            \includegraphics[width=\textwidth]{plaatjes/slanted-plate}
            \caption{Slanted characters.}
            \label{fig:slanted-plate}
        \end{subfigure}%
        ~ 
        \begin{subfigure}{0.33\textwidth}
            \includegraphics[width=\textwidth]{plaatjes/unslanted-plate}
            \caption{Unslanted characters.}
            \label{fig:unslanted-plate}
        \end{subfigure}%

        \begin{subfigure}{0.33\textwidth}
            \includegraphics[width=\textwidth]{plaatjes/plate-no-slant}
            \caption{Not slanted characters.}
            \label{fig:not-slanted-plate}
        \end{subfigure}%
        ~ 
        \begin{subfigure}{0.33\textwidth}
            \includegraphics[width=\textwidth]{plaatjes/unslanted-incorrectly}
            \caption{Incorrectly unslanted characters.}
            \label{fig:incorrectly-unslanted-plate}
        \end{subfigure}%

        \caption{Process to unslant characters.}
        \label{fig:unslanting-plate}
\end{figure}

\paragraaf{Text recognition}

One way to solve the problem of similar shape characters being wrongly recognised is to create a dictionary where similar characters are grouped together. Every time one of these characters is recognised, the other characters in the group can be given as a possible alternative and a confidence percentage can be calculated according to which characters are present in the text.